\section{Event selection and inclusive higgs decay measurement}\label{sec:event_selection}
%Two steps of analysis are taken. In the first step, a series of object and event selection was applied to maximize the sensitivity to detect the signal; in the second step,  a template fit on the 2-Dimension flavor weight distribution, which is got from TMVA based flavor tagging algorithm, is implemented to further distiguish the flavor components in final states. The details of template fit are described in \ref{sec:templatefit}.\par
%The final states with 2 jets, 2 jets + 2 leptons(either electron or muon pair), 4 jets are required for \nnh,\llh and \qqh channel respectively.
%\par
Three steps of analysis are taken. In the first step, an inclusive study of $ZH$ production in $\llh$ channel was done. A series of cuts was applied to the lepton pairs. The $ZH$ events number are extracted by fitting the lepton pair recoil mass spectrum. In the second step, the cuts on jets information are applied, to further suppress the backgrounds. In the last stage, a fit was applied on both the flavor tagging information and lepton pair recoil mass, 
to extract the events number of $ZH\to l\bar{l}+\bpair/\cpair/\gpair$. 
The higgs decay branch ratio was calculated by dividing the individual $ZH\to l\bar{l}+\bpair/\cpair/\gpair$  number by the inclusive $ZH\to \llh$ number.

\input{texfiles/EventSelection/llh_eventselection.tex}
\subsection{Inclusive higgs measurement}
\label{sec:inclusive_higgs_measurement}
Once the lepton pair was selected, we can extract the inclusive higgs decay number. 
This procedure is similar to what is done in \cite{CEPC:recoilmass}. 
After applying the cuts on azimuth angle and invariant mass of lepton pair system, the $ZH$ events number are worked out with a fit on the muon recoil system's mass. 
In the fit, the recoil masses shape of $ZH$ events and other SM processes are described by a crystal ball function and a second order Chebychev polynomial function respectively. 
The fit range was set as $110\GeV<M_{recoil}<140$\GeV.  \par
%\input{texfiles/EventSelection/nnh_eventselection.tex}
%\input{texfiles/EventSelection/qqh_eventselection.tex}
