\section{Introduction}\label{sec:introduction}
The discovery of a scalar boson with mass around 125 \GeV at LHC \cite{Higgs_ATLAS, Higgs_CMS} completed the final piece of the standard model.
This particle, interpreted as the Higgs boson, plays a crucial role in the electroweak spontaneous symmetry broken (EWSB), 
known as the BEH (Brout-Englert-Higgs) mechanism \cite{BEH,BEH2,BEH3}, often referred as the Higgs mechanism.
The higgs meachism allows the $\Wboson$, $\Zboson$, 
quarks and charged leptons to be massive while keeping the $SU(2)_L \times U(1)_Y$ gauge invariance.
The Higgs mechanism allows the $\Wboson$, $\Zboson$, quarks and charged leptons to be massive while keeping the $SU(2)_L \times U(1)_Y$ gauge invariance. The masses of the SM fermions ($m_{f_i}$) in the SM are proportional to their Yukawa couplings ($h_i$) to the Higgs field: $m_{f_i} = \dfrac{vh_i}{\sqrt{2}}$, where $v\approx 246\GeV$ is the vacuum expectation value of the Higgs field (VEV).

Measuring the Yukawa couplings between higgs and SM fermions is essential to undertand the origin of the fermions' masses and the detail of EWSB.
The dominant direct higgs decays into fermionic final states are $\Hboson\to \bpair$, $\Hboson\to\cpair$ and $\Hboson\to\taupair$, 
the decay branching ratios of which are estimated to be \percent{57}, \percent{4} and \percent{3}. 
In addition, the Higgs boson can decay to gluon pairs via a heavy quark loops. The large coupling between higgs and top quark leads to considerably large 
branching ratio of $\Hboson\to\gpair$ that is estimated to be about \percent{9}. 
\par
Until now, the LHC is the collider to directly study the Higgs mechansim. 
The leading higgs fermionic decay,  $\Hboson\to\bpair$ was studied in both ATLAS and CMS experiment in VH\cite{VH_bb_atlas,VH_bb_cms}, ttH\cite{ttH_bb_cms} and VBF\cite{VBF_bb_atlas,VBF_bb_cms} process, with the LHC Run-I data.
The combination of ATLAS and CMS gives $\bpair$ $\sigma\times Br$ signal strength for $0.70\pm0.29$ in run-I data\cite{higgs_atlas_cms_combine}. 
A study on $\Hboson\to\bpair$ in which Higgs are produced in association with $W$ or $Z$ boson, using \ifb{36.1} Run-II data with center of mass 13 TeV in ATLAS, 
constrained the signal strength as $1.20^{0.24}_{-0.23}(stat.)^{+0.34}_{-0.28}(sys.)$ \cite{VHbb_atlas_run2} with 3.5 $\sigma$ signal significance.
The large uncertainty is due to huge QCD or vector boson production with muliti-jets backgrounds, which is inevitable in hadron colliders. 
\par
%The Circular Electron Positron Collider(CEPC) \cite{CEPC_preCDR} program is proposed with the goal to better understand the EWSB
% by precisely measuring on these higgs parameters as well as other EW parameters of interest. 
% The CEPC has the advantages in precision measurement:
% \begin{itemize}
% \item Clean backgrounds
% \item Well defined frame of center momentum
% \item High luminosity
% \end{itemize}
On the other hand, with much fewer background events and well defined initial and final states, the proposed Circular Electron Positron Collider (CEPC)~\cite{CEPC_preCDR} 
will have capability to carry out precision measurements of many Higgs decay modes that are inaccessible at the LHC.
% The center of mass enerey is configured as \GeV{250}, well above the $\Zboson\Hboson$ threshold.
% With the nominal luminosity at \ten{2}{34}\cm2s1, \ifb{5000} data can be accumulated in ten years of running. 
% Over one million higgs boson will be produced. Clean background and high statistics make precise measurement possible.
%The outline of higgs background will be discussed in section \ref{sec:CEPC_MC}
%\par
%The works presented in this note demostrate the capability of the $\Hboson \to \bpair/\cpair/\gpair$ measurements in CEPC.
%The higgs productions associate with charged lepton(electrons or muons) pair, neutrino pair or quark pair are studied.
% In Section II, a brief introduction of CEPC experiment and the MC sample will be presented. In section III the event selection and the analysis strategy will be described. In section IV the results are listed and discussed. Detail information, auxiliary figures, tables and numbers, as well as analysis method in study can be found in appendix.
%\par
The studies presented in this note demonstrate the capability to perform precision measurements of the Higgs boson decay branching ratio in the $\bpair$,$\cpair$ and $\gpair$ final states 
at the CEPC. We consider the Higgs boson production mainly in association with a $Z$ boson, where $Z$ is reconstructed in the $\nu\bar{\nu}$, $ell\ell$ and $\qpair$ final states.
This note is organized as follows: section II gives a brief introduction of the CEPC and MC samples used in the analysis; section III focuses on the analysis detail including event selection 
of the $\llh$, $\nnh$ and $\qqh$ channel; in section IV, flavor tagging issues will be described, including the tagging techniques and template fit method; in section V, the validation to 
the template fit will be discussed; in section VI and VII, the results and conclusion will be presented.
\clearpage
