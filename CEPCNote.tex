%%%%%%%%%%%%%%%%%%%%%%%%%%%%%%%%%%%%%%%%%%%%%%%%%%%%%%%%%%%%%%%%%%%%%%%%%%%%%
%
% This is a template CEPC Paper that contains suggestions and hints on
% how to get your note in a form that minimizes the amount of work
% needed to get it approved by the collaboration - assuming that the
% physics is OK!
%
%%%%%%%%%%%%%%%%%%%%%%%%%%%%%%%%%%%%%%%%%%%%%%%%%%%%%%%%%%%%%%%%%%%%%%%%%%%%%%

%\documentclass[11pt,a4paper]{cepcnote}
\documentclass[coverpage]{cepcnote} 
\graphicspath{{figures/}}
\usepackage{cepcphysics}
\usepackage{subfigure}
\usepackage{mathrsfs}
\usepackage{authblk}
\usepackage{amsmath, amssymb}
\usepackage{slashed}
\usepackage{multirow}
\usepackage{lineno}
%\newcommand{\bslash}{\ensuremath{\backslash}}
\def\Slash#1{\!\not\!\! #1}             % Feynman dagger for upper case
\def\slash#1{\!\not\! #1}               % Feynman dagger for lower case
\newcommand{\etmiss}{\ensuremath{\slashed{E}_T}} %added by baiyu
\newcommand{\emiss}{\ensuremath{\slashed{E}}}    %added by baiyu
\newcommand{\mmiss}{\ensuremath{\slashed{M}}}    %added by baiyu
\newcommand{\mmissquare}{\ensuremath{\slashed{M}^2}} %added by baiyu

\newcommand{\minitab}[2][l]{\begin{tabular}{#1}#2\end{tabular}}  %added by baiyu
\renewcommand{\multirowsetup}{\centering}       %added by baiyu
\newcommand{\tabincell}[2]{\begin{tabular}{@{}#1@{}}#2\end{tabular}} %added by baiyu

%font size
\newcommand{\zhonghao}{\fontsize{10pt}{\baselineskip}\selectfont}
\newcommand{\chuhao}{\fontsize{8pt}{\baselineskip}\selectfont}
\newcommand{\xiaochuhao}{\fontsize{5.4pt}{\baselineskip}\selectfont}
\newcommand{\xiaohao}{\fontsize{5.0pt}{\baselineskip}\selectfont}

%%% commands Alex
\newcommand{\inmath}[1]{\ensuremath{#1}\xspace}
%\newcommand{\eg}{e.\,g.\@\xspace}
%\newcommand{\Eg}{E.\,g.\@\xspace}
%\newcommand{\ie}{ie\@ifnextchar.{}{.\@}}
%\newcommand{\Ie}{I.\,e.\@\xspace}
\newcommand{\percent}[1]{{#1}\,\%}
\newcommand{\todo}[1]{{\color{red}TODO: #1}}
\newcommand{\metx}{\inmath{\met}}
\newcommand{\meffx}{\inmath{m_\text{eff}}}
\newcommand{\mttwox}{\inmath{m_\text{T2}}}
\newcommand{\summtx}{\inmath{m_T(\lepton_1) + m_T(\lepton_2)}}
\newcommand{\mctx}{\inmath{m_\text{CT}}}
\newcommand{\Wjetsx}{$W$~+ jets\xspace}
\newcommand{\Zjetsx}{$Z$~+ jets\xspace}
\newcommand{\ttbarx}{\ttbar\xspace}
\newcommand{\ptx}{\pt\xspace}
\newcommand{\Table}[1]{Table~\ref{#1}\xspace}
\newcommand{\Fig}[1]{Figure~\ref{#1}\xspace}
%\newcommand{\GeV}[1]{\unit[#1]{GeV}\xspace} % I made sure no one is using \GeV before adding this line
\newcommand{\ifbx}[1]{\unit[#1]{\ifb}\xspace}
\newcommand{\eighttev}{\inmath{\sqrt{s} = \unit[8]{TeV}}}
\newcommand{\com}{\inmath{\sqrt{s} = \unit[250]{GeV}}}
\newcommand{\lumiunit}{\ensuremath{\mathrm{cm}^{-2}\mathrm{s}^{-1}}
}
\makeatletter
\DeclareRobustCommand*{\trigger}[1]{%
  \begingroup\@activeus\scantokens{#1\endinput}\endgroup}
\begingroup\lccode`\~=`\_\relax
   \lowercase{\endgroup\def\@activeus{\catcode`\_=\active \let~\_}}
\makeatother
\newcommand{\neutralinon  }[1]{\inmath{\widetilde{\chi}_{#1}^0}}
\newcommand{\charginon    }[1]{\inmath{\widetilde{\chi}_{#1}^\pm}}
\newcommand{\taupll       }{tau~+ light lepton\xspace}
\newcommand{\lepton       }{\ensuremath{\ell}}
\newcommand{\nnh          }{\ensuremath{\nu \bar{\nu} H} }
\newcommand{\llh          }{\ensuremath{l^{+}l^{-}H}}
\newcommand{\eeh          }{\ensuremath{e^{+}e^{-}H}}
\newcommand{\mmh          }{\ensuremath{\mu^{+}\mu^{-}H}}
\newcommand{\tautauh      }{\ensuremath{\tau^{+}\tau^{-}H}}
\newcommand{\qqh          }{\ensuremath{q\bar{q}H}}
\newcommand{\bpair           }{\ensuremath{b\bar{b}}}
\newcommand{\cpair           }{\ensuremath{c\bar{c}}}
\newcommand{\gpair           }{\ensuremath{gg}}
\newcommand{\qpair           }{\ensuremath{q\bar{q}}}
\newcommand{\leppair           }{\ensuremath{l^{+}l^{-}}}
\newcommand{\leppm             }{\ensuremath{l^{\pm}}}
\newcommand{\lepmp             }{\ensuremath{l^{\mp}}}
\newcommand{\elpair           }{\ensuremath{e^{+}e^{-}}}
\newcommand{\elpm             }{\ensuremath{e^{\pm}}}
\newcommand{\elmp             }{\ensuremath{e^{\mp}}}
\newcommand{\mupair           }{\ensuremath{\mu^{+}\mu^{-}}}
\newcommand{\mupm             }{\ensuremath{\mu^{\pm}}}
\newcommand{\mump             }{\ensuremath{\mu^{\mp}}} 
\newcommand{\taupair           }{\ensuremath{\tau^{+}\tau^{-}}}
\newcommand{\taupm             }{\ensuremath{\tau^{\pm}}}
\newcommand{\taump             }{\ensuremath{\tau^{\mp}}}
%\newcommand{\ten          }[1]{\cdot 10^{#1}}
\newcommand{\ten           }[2]{\ensuremath{{#1}\times 10^{#2}}}
%%%

% Shorthand for \phantom to use in tables
\newcommand{\pho}{\phantom{0}}
\newcommand{\bslash}{\ensuremath{\backslash}}
\newcommand{\BibTeX}{{\sc Bib\TeX}}

%%%%%%%%%%%%%%%%%%%%%%%%%%%%%%%%%%%%%%%%%%%%%%%%%%%%%%%%%%%%%%%%%%%%%%%%%%%%%%
% Preamble
%%%%%%%%%%%%%%%%%%%%%%%%%%%%%%%%%%%%%%%%%%%%%%%%%%%%%%%%%%%%%%%%%%%%%%%%%%%%%%

%\title{ Higgs Mass and Cross-section Measurement at CEPC }
%\title{Measurement of $H\to \bpair /\cpair /\gpair$ Branch Ration from $ZH$ Production with  $\nnh$, $\llh$ and $\qqh$  Final States in CEPC Experiment} 
\title{Measurements of the decay branching ratios of $H\to\bpair/\cpair/\gpair$ at CEPC}
%\title{A template for CEPC papers}
\author{CEPC Workgroup}
\mail  {baiy@seu.edu.cn}
%\draftversion{1.0}
\cepcnote{CEPC\_ANA\_HIG\_2016\_XXX}
\abstracttext{
  
%  This is a template CEPC paper. It contains the structure, style
%  files and hints on how to produce a paper for which a minimum amount
%  of time is necessary to spend on typographic details. This template
%  can be found on the web pages of the CEPC Collaboration. 
%  You can find some \LaTeX{} technical detail about the
%  template in the Appendix of this paper.
%  A couple of remarks about the paper front page:
%  \begin{itemize}
%
%  \item {\bf Title:} Measurement of $Hboson\to \bpair /\cpair /\gpair$ Branch Ration from $ZH$ Production with \nnh, \llh and \qqh  Final States in CEPC Experiment
%  \item {\bf Author list:} it will be provided by the CEPC Collaboration,
%    and will be made available on their website. On the
%    front page, you should name ``The CEPC Collaboration'' as
%    author.

%  \item {\bf Abstract:}
  {\bf Abstract:}
%  A study on the measurement on $H\to \bbbar /\ccbar/gg$ branch ratio, one of the benchmark measurements in CEPC experiment, is presented here. In the scenario concerned, the Higgs boson are produced associated with a Z boson,
%  and subsequently undergo hadronic decay, while the $Zboson$ decays to neutrino pair, charged lepton pair or quark pair.  
%
%%  This analysis,in complementary with the measurement on other states of ZH production, can be used to derive the Yakawa coupling, one of the key properties to understand the Electroweak symmetry breaking mechanism. \par
%   An integral luminosity of 5000 fb$^{-1}$ is assumed to estimate the signal and background yielding, corresponding to 10 years running of CEPC with nominal luminosity at \ten{2}{34} $\lumiunit$. A cut based analysis is applied to each analysis, combined with multi-variable method. The flavor information in final states 
%   are extracted by fitting each flavor components according to their templates. A Toy Monte-Carlo test was done to evaluate the statistic uncertainty. 
%   They systematic uncertainty was also disccussed and estimated.
%   We conclude the measurement of $\Hboson \to \bpair/\cpair/\gpair$ reach the precision \percent{0.2}, \percent{2} and \percent{3}. 
   %A cut based analysis gives xxx signal events and xxx.xxx background events. The flavor components was distinguished using a template fit method. In addition study of various of technics such as TMVA, tag counting are on going.
%    it should also be clear, descriptive, and
%    concise. It should ideally be one paragraph long, and certainly no
%    more than half a page. It should stand on its own and, similarly,
%    the main text of the paper should not depend on it. The abstract
%    should state: what was the measurement; where was it done and with
%    what dataset/luminosity; what method was used; what are the
%    primary results and main conclusions.  Citations in an abstract
%    should be avoided. If only Monte Carlo data are used in the
%    publication, this fact should be stated explicitly in the
%    abstract.  


   A study on the measurements of $H\to \bbbar /\ccbar/gg$ decay branching fractions is presented in this section.
Here we consider the Higgs boson production mainly in association with a $Z$ boson, where $Z$ is reconstructed in the \mupair final states. 
An integrated luminosity of 5000 fb$^{-1}$ is assumed in the analysis, corresponding to 10 years data taking with a nominal luminosity of \ten{2}{34} $\lumiunit$ at the CEPC. 

%The results presented here are based on a combination of cut based event selection and a multi-variable template fitting method.  
%The expected statistical uncertainties of the measurements are estimated using toy Monte-Carlo tests. 

%Several potentially dominant systematic uncertainties of the measurements are also evaluated. The study shows that the branching fraction measurements of the Higgs boson decaying to $\bpair$, $\cpair/$ and $\gpair$ can reach a precision of \percent{0.2}, \percent{2} and \percent{3}, respectively   
%  \end{itemize}
}

%%%%%%%%%%%%%%%%%%%%%%%%%%%%%%%%%%%%%%%%%%%%%%%%%%%%%%%%%%%%%%%%%%%%%%%%%%%%%%%
% This is where the document really begins
%%%%%%%%%%%%%%%%%%%%%%%%%%%%%%%%%%%%%%%%%%%%%%%%%%%%%%%%%%%%%%%%%%%%%%%%%%%%%%%

\begin{document}
\linenumbers
\tableofcontents
\clearpage

\section{Introduction}\label{sec:introduction}
The discovery of a scalar boson with mass around 125 \GeV at LHC \cite{Higgs_ATLAS, Higgs_CMS} completed the final piece of the standard model.
This particle, interpreted as the Higgs boson, plays a crucial role in the electroweak spontaneous symmetry broken (EWSB), 
known as the BEH (Brout-Englert-Higgs) mechanism \cite{BEH,BEH2,BEH3}, often referred as the Higgs mechanism.
The higgs meachism allows the $\Wboson$, $\Zboson$, 
quarks and charged leptons to be massive while keeping the $SU(2)_L \times U(1)_Y$ gauge invariance.
The Higgs mechanism allows the $\Wboson$, $\Zboson$, quarks and charged leptons to be massive while keeping the $SU(2)_L \times U(1)_Y$ gauge invariance. The masses of the SM fermions ($m_{f_i}$) in the SM are proportional to their Yukawa couplings ($h_i$) to the Higgs field: $m_{f_i} = \dfrac{vh_i}{\sqrt{2}}$, where $v\approx 246\GeV$ is the vacuum expectation value of the Higgs field (VEV).

Measuring the Yukawa couplings between higgs and SM fermions is essential to undertand the origin of the fermions' masses and the detail of EWSB.
The dominant direct higgs decays into fermionic final states are $\Hboson\to \bpair$, $\Hboson\to\cpair$ and $\Hboson\to\taupair$, 
the decay branching ratios of which are estimated to be \percent{57}, \percent{4} and \percent{3}. 
In addition, the Higgs boson can decay to gluon pairs via a heavy quark loops. The large coupling between higgs and top quark leads to considerably large 
branching ratio of $\Hboson\to\gpair$ that is estimated to be about \percent{9}. 
\par
Until now, the LHC is the collider to directly study the Higgs mechansim. 
The leading higgs fermionic decay,  $\Hboson\to\bpair$ was studied in both ATLAS and CMS experiment in VH\cite{VH_bb_atlas,VH_bb_cms}, ttH\cite{ttH_bb_cms} and VBF\cite{VBF_bb_atlas,VBF_bb_cms} process, with the LHC Run-I data.
The combination of ATLAS and CMS gives $\bpair$ $\sigma\times Br$ signal strength for $0.70\pm0.29$ in run-I data\cite{higgs_atlas_cms_combine}. 
A study on $\Hboson\to\bpair$ in which Higgs are produced in association with $W$ or $Z$ boson, using \ifb{36.1} Run-II data with center of mass 13 TeV in ATLAS, 
constrained the signal strength as $1.20^{0.24}_{-0.23}(stat.)^{+0.34}_{-0.28}(sys.)$ \cite{VHbb_atlas_run2} with 3.5 $\sigma$ signal significance.
The large uncertainty is due to huge QCD or vector boson production with muliti-jets backgrounds, which is inevitable in hadron colliders. 
\par
%The Circular Electron Positron Collider(CEPC) \cite{CEPC_preCDR} program is proposed with the goal to better understand the EWSB
% by precisely measuring on these higgs parameters as well as other EW parameters of interest. 
% The CEPC has the advantages in precision measurement:
% \begin{itemize}
% \item Clean backgrounds
% \item Well defined frame of center momentum
% \item High luminosity
% \end{itemize}
On the other hand, with much fewer background events and well defined initial and final states, the proposed Circular Electron Positron Collider (CEPC)~\cite{CEPC_preCDR} 
will have capability to carry out precision measurements of many Higgs decay modes that are inaccessible at the LHC.
% The center of mass enerey is configured as \GeV{250}, well above the $\Zboson\Hboson$ threshold.
% With the nominal luminosity at \ten{2}{34}\cm2s1, \ifb{5000} data can be accumulated in ten years of running. 
% Over one million higgs boson will be produced. Clean background and high statistics make precise measurement possible.
%The outline of higgs background will be discussed in section \ref{sec:CEPC_MC}
%\par
%The works presented in this note demostrate the capability of the $\Hboson \to \bpair/\cpair/\gpair$ measurements in CEPC.
%The higgs productions associate with charged lepton(electrons or muons) pair, neutrino pair or quark pair are studied.
% In Section II, a brief introduction of CEPC experiment and the MC sample will be presented. In section III the event selection and the analysis strategy will be described. In section IV the results are listed and discussed. Detail information, auxiliary figures, tables and numbers, as well as analysis method in study can be found in appendix.
%\par
The studies presented in this note demonstrate the capability to perform precision measurements of the Higgs boson decay branching ratio in the $\bpair$,$\cpair$ and $\gpair$ final states 
at the CEPC. We consider the Higgs boson production mainly in association with a $Z$ boson, where $Z$ is reconstructed in the $\nu\bar{\nu}$, $ell\ell$ and $\qpair$ final states.
This note is organized as follows: section II gives a brief introduction of the CEPC and MC samples used in the analysis; section III focuses on the analysis detail including event selection 
of the $\llh$, $\nnh$ and $\qqh$ channel; in section IV, flavor tagging issues will be described, including the tagging techniques and template fit method; in section V, the validation to 
the template fit will be discussed; in section VI and VII, the results and conclusion will be presented.
\clearpage

\section{CEPC Experiment and MC Sample}
\label{sec:CEPC}
\subsection{CEPC experiment}
The CEPC is a future ciruclar electron-positron collider project. Two detectors will be installed at two interaction 
points in the stoarge ring, 50-100 kilometers in circumference. 
To study the Higgs boson property, electrons and positrons collide at each interaction point with center of mass energy 240 - 250 GeV. The luminosity is design to be \ten{2}{34} \lumiunit. 
The CEPC can deliver up to 5000 \ifb data after ten years of running , including about 1 million Higgs events.
 The Higgs boson are produced mainly via associated production with Z boson(96.6\%) as well as much $WW$ fusion (with $\nu_e\bar{\nu}_e H$ in final states,3.06\%) and $ZZ$ fusion(0.29\%, with $e^+e^-H$ in final state).
 

 ILD-like detector is designed as the CEPC detector(CEPC-v1) with additional considerations\cite{CEPC_preCDR}.  
%A machine detector interface (MDI) was designed at the interaction region(IR). It covers the common aspects in the collider and the detector. 
A vertex detector (VTX) with high pixel resolution is located in the inner 
most part of the detector. It provides the inner tracks measurement with high spatial resolution, which is key information for track impact parameter 
 (IP) measurements and vertex reconstruction. The heavy flavor jet tagging significantly relies on the capablity on IP measurements and vertex reconstruction with high precision. 
%This sub-detector is crucial for the performance of flavor tagging. 
The 6 layers of sensors are laid 16 to 60 mm in radius with 97\% to 90\% in azimuth angular coverage. The signal layer spatial resolution is 2.8 $\mu$m in the 2 inner 
layers and 4 $\mu$m in the 4 outer layers. The overall IP resolution can be estimated as 
\begin{equation}
{\sigma(r\phi)}= a \oplus \frac{b}{p(\mathrm{\GeV}) \sin^{3/2} \theta }\mum
%\sigma_{SP} = a \bigotimes \frac{b}{p(\mathrm{GeV})\Sin^{3/2}\theta \mu\mathrm{m}}
%\sigma_{SP} = a \bigotimes \frac{b}{c}
\end{equation},
in which $a = 5$ and $b = 10$.\par %how about the logitudinal impact parameter
A Time Project Chamber(TPC) is laid out of VTX to take the major task of track 
measurement. It covers the solid angle up to $\cos \theta = 0.98$. 
When being operated in 3.5 T filed, the momentum resolution is 
$\sigma(1/p_T) = 10^{-4}\GeV$. \par
A Particle Flow Algorithm-oriented\cite{PFA} calorimeter system, combined by the electromagnetic calorimeter(ECAL) 
and hadronic calorimeter(HCAL), was designed with high granularity and precise energy mesurement of electrons, photons, taus and hadronic jets. 
The resolution of ECAL and HCAL are about $16\%/\sqrt{E(\GeV)}$ and $50\%/\sqrt{E(\GeV)}$. The energy resolution of jets from Higgs or $W^{\pm}/Z$ 
decay is estimated about $\sigma_E/E = 3-4 \%$. With granularity better than 1 \cm $\times$ 1 \cm of each cell, the hadrons in jets can be well seperated. \par
The muon system is mounted as the outermost part in the detector. The baseline 
design of muon detector requires 94\% reconstruction efficiency of muons with 
energy higher than 6 \GeV. The longitutute and transverse position resolution 
are required to be $\sigma _{z} = 1.5 \cm$ and $\sigma_{r\phi} = 2.0 \cm$. The 
rate of pions mis-identified as muons at energy 30 \GeV is required to be less 
than 1\%.\par

%\subsection{CEPC detector}
%A ILD-like detector is designed as the CEPC detector(CEPC-v1) with additional considerations\cite{CEPC_preCDR}. Detail description of ILD model can be found in \cite{ILD_detector}. All changes need to be implemented into simulation, and iterate with physics analysis and cost estimation.

\subsection{MC Samples and Jets Reconstruction}
In this analysis, the signal events are $e^+e^-\to \Zzero \Hboson\to \leppair + \bbbar/\ccbar/gg$, thus the final states contains a pair of leptons with opposite charge and two jets.  
The standard model background includes di-quark events, di-lepton events, vector boson pair production and higgs production with final states different from the signal. Both background and signal events are generated using Whizard\cite{Wizard_1} configured as no-polarization electron-positron collision at center of mass energy of 250 GeV. PHYTHIA 6.4 \cite{PYTHIA64} was used to model the fragmentation and hadronization. The higgs mass was assumed to be 125 GeV and the coupling was set as that predicted by standard model. Detailed information of MC samples generation can be found in \cite{Samples}.\par
The generated events undergo the GEANT4\cite{Geant4} based detector simulator Mokka\cite{mokka} with CEPC-v1. The simulated hits were digitized and reconstructed with ArborPFA\cite{ArborPFA}. \par
%The  LCFIPlus\cite{LCFIPlus} toolkit was utilized to reconstruct the primary vertex, jets and secondary vertex. 
Jets reconstruction and flavor tagging are essential to this analysis.
These tasks are done with the lcfiplus \cite{LCFIPlus} toolkit, integrating the functionality of doing vertex finding, jet reconstruction and jet flavor tagging. 
Flavor tagging will be discussed in section \ref{subsec:flavortagging}. 
Jets are reconstructed by Durham algorithm\cite{Durham}. 
% by implementing the lcfiplus\cite{LCFIPlus} toolkit. 
This algorithm begins with jet cluster candidates and a required jet multiplicity. 
Initially each track from primary vertex or each neutral particle is a jet cluster candidates. Tracks from the same secondary vertex are taken as one jet cluster candidates.
The procedure iteratively pairs the clusters by selecting the minimum 
distance measure, defined as $y_{i,j} = \mathrm{min}\{E^2_i,E^2_j\}(1-\cos(\theta_{ij})/E^2_{vis}$, where $E_i$ and $E_j$ are the energy of 
$i$-th and $j$-th cluster, and $\theta_{ij}$ refers to the angle between them. 
$E_{vis}$ are the sum of energy of all the clusters. Clusters with minimum $y_{ij}$ are merged, reducing the cluster number by 1,until the remaining clusters number 
equals to the required jet multiplicity. This algorithm gives a series of 
$y_{ij}$ value: $y_{12}$, $y_{23}$, $y_{34}$ $etc$. When $i$ is larger than 
the real cluster multiplicity, $y_{ij}$ refect the distance of two clusters 
which are actually from the same cluster, resulting a small $y_{ij}$ value. Thus $y_{ij}$ is an indicator of the jet cluster multiplicity.\par
%This algorithm is integrated in the 
%lcfiplus \cite{LCFIPlus} toolkit. 
%This toolkit is also capable to reconstruct the primary and secondary vertex, as well as jet flavor tagging.\par
%Background events without higgs production undergo fast simulation, which includes:
%\begin{itemize}
%\item Four momentum of jet(b,c quarks and gluon) is smeared according to a Gaussian function, with jet energy  
%resolution $\sigma$ set to be 4\%.
%\item Each leptonic track ($e/\mu$) is corrected by momentum resolution and tracking efficiency, whose parameter are obtained from the study of the full simulation.
%\item The four momentum of neutrino decaying from the final hadron subtracted from the four momentum of jet.
%\end{itemize} 
%The fast simulation reconstruct jets using a simplified algorithm. It is faster but with disadvantage that the vertex information is missing from reconstruct. Full simulation sample of the backgrounds are also in production and results will be updated using those samples. 
\clearpage

\section{Event selection and inclusive higgs decay measurement}\label{sec:event_selection}
%Two steps of analysis are taken. In the first step, a series of object and event selection was applied to maximize the sensitivity to detect the signal; in the second step,  a template fit on the 2-Dimension flavor weight distribution, which is got from TMVA based flavor tagging algorithm, is implemented to further distiguish the flavor components in final states. The details of template fit are described in \ref{sec:templatefit}.\par
%The final states with 2 jets, 2 jets + 2 leptons(either electron or muon pair), 4 jets are required for \nnh,\llh and \qqh channel respectively.
%\par
Three steps of analysis are taken. In the first step, an inclusive study of $ZH$ production in $\llh$ channel was done. A series of cuts was applied to the lepton pairs. The $ZH$ events number are extracted by fitting the lepton pair recoil mass spectrum. In the second step, the cuts on jets information are applied, to further suppress the backgrounds. In the last stage, a fit was applied on both the flavor tagging information and lepton pair recoil mass, 
to extract the events number of $ZH\to l\bar{l}+\bpair/\cpair/\gpair$. 
The higgs decay branch ratio was calculated by dividing the individual $ZH\to l\bar{l}+\bpair/\cpair/\gpair$  number by the inclusive $ZH\to \llh$ number.

\input{texfiles/EventSelection/llh_eventselection.tex}
\subsection{Inclusive higgs measurement}
\label{sec:inclusive_higgs_measurement}
Once the lepton pair was selected, we can extract the inclusive higgs decay number. 
This procedure is similar to what is done in \cite{CEPC:recoilmass}. 
After applying the cuts on azimuth angle and invariant mass of lepton pair system, the $ZH$ events number are worked out with a fit on the muon recoil system's mass. 
In the fit, the recoil masses shape of $ZH$ events and other SM processes are described by a crystal ball function and a second order Chebychev polynomial function respectively. 
The fit range was set as $110\GeV<M_{recoil}<140$\GeV.  \par
%\input{texfiles/EventSelection/nnh_eventselection.tex}
%\input{texfiles/EventSelection/qqh_eventselection.tex}

\input{texfiles/templatefit}
%\input{texfiles/validation}
%\input{texfiles/results}
\section{Uncertainties of Measurements}
The statistic uncertainty was estimated by applying the toyMC method. 
These procedure includes 5000 iterations. In each iteration, the 'data' is filled in a 3D histogram in $X_B$-$X_C$-$recoil mass$ space. 
Then, in each bin of the histogram, the event yields fluctated according to a Poisson distribution. 
The 3D fit is applied to the fluctated 'data'. The dispersion of fitted signal event yields in signal region represents the statistic uncertainty. 
The results of toyMC test for $H\to\bpair$, $H\to\cpair$ and $H\to gg$ are represented in Fig \ref{fig:toyMC}, in terms of standard deviation of fitted signal strength. The statistic 
uncertainty of fitted signal strength for $H\to\bpair$, $H\to\cpair$ and $H\to gg$ are estimated to be 1.11\%, 10.5\% and 5.44\% respectively.
The statistic uncertainty of inclusive higgs decay is estimated in similar way, 
except for that he toyMC sample was generated by fluctuate lepton one-dimension recoil mass sepctrum, instead of three-dimension flavor-recoil mass distribution.
\begin{figure}
\label{fig:toyMC}
\centering
\subfigure[]
{ 
   \begin{minipage}[b]{0.31\textwidth}
   \includegraphics[width=\textwidth]{Template/toymc_mumuh_bb.pdf}
   \end{minipage}
}
\subfigure[]
{
   \begin{minipage}[b]{0.31\textwidth}
   \includegraphics[width=\textwidth]{Template/toymc_mumuh_cc.pdf}
   \end{minipage}
}
\subfigure[]
{
   \begin{minipage}[b]{0.31\textwidth}
   \includegraphics[width=\textwidth]{Template/toymc_mumuh_gg.pdf}
   \end{minipage}
}
\caption{Toy MC test result in terms of signal strength and uncertainty in signal region from template fit.}
\end{figure}

\par
Since the we are interested in $\dfrac{N_{H\to\bpair\cpair\gpair}}{N_{H,inclusive}}$, systematic uncertainties from lepton pair invariant mass cut, recoil 
masses cut, $Z\to\mupair$ modeling, and ISR correction canceled. Meanwhile, the signal events yields are worked out in model-indepdent way so that uncertainty from MC generator can be ignored. 
The remaining systematic sources include uncertainty from fit method, uncertainty of jet selection efficiencies, uncertainty from flavor templates and uncertainty due to non-uniformity 
in isolation lepton selection efficiency.\par
The fit method have two types of systematic uncertainty. The first kind of uncertainty is due to imperfect modeling of the PDFs. 
The PDFs include the recoil mass modeling and flavor template. The latter will be 
discussed as the systematic of flavor tagging and here we only focus on the 
recoil mass modeling.
The toyMC samples used in statistic uncertainty study are used to estimate this type of systematic uncertainty. 
The difference between the central value of fitted signal strength in toyMC test from the MC prediction are taken as the systematic uncertainty due to the modeling.\par
%By looking for the bias of fitted 
%signal events yields from that of MC prediction, we estimate the systematic uncertainty from the models in fit is 1.07\% for $H\to\bpair$, for $H\to\cpair$, for $H\to\gpair$ in $\mmh$ channel, while in $\eeh$ channel .\par
The other kind of systematic uncertainty in the fit comes from the uncertainty of we fixed parameters.
 The normalization parameters for $H\to WW^*$ 
 and $H\to ZZ^*$ backgrounds are fixed in the fit. We set these normalization parameters  5\% higher and lower to find their impact on the fitted signal yields. 
 We conservatively vary the yields of backgrounds other than $\llh$ and $ZZ\to\leppair+\qpair$ by 100\% to estimate the systematic uncertainty of fixing them.  
\par
Systematic uncertainty on lepton veto efficiency for $H\to\bpair$ can be estimate in $Z-pole$ $\bpair$ events. With one jet tagged as $b-jets$. The efficiency of lepton veto can be studied with the precision of 0.0002\% with 2 billion \bpair events in the $Z$-pole sample.
. Thus we take 0.002\% as the systematic uncertainty on lepton veto. For $H\to\cpair$ with similar method the uncertainty of lepton veto efficiency is estimated as 0.0001\%. For $H\to\gpair$ we assume the gluon jets sample yielding 1\% of that of \bpair, and estimate the 
uncertainty of lepton veto to be 0.008\%.  
The systematic uncertainties of jet PFO multiplicity, jet $\cos \theta$ cut 
and $y-$th value cut was estimated in similar way, such that by assuming these variables can be 
calibrated with the $Z-$pole sample. \par
The systematic uncertainty of the efficiency on jet pair invariant mass cut can be 
estimated from the jet energy resolution. We apply a smearing on jet pair mass distribution according to a guassian distribution corresponding to the jet energy resolution. 
We take the value of 4\% as the jet energy resolution from CEPC pre-CDR\cite{CEPC_preCDR} and get the uncertainty for $H\to \bpair$, $H\to\cpair$ and $H\to\gpair$ are $^{+0.68\%}_{-0.20\%}$, $^{+0.43\%}_{-1.08\%}$ and $^{+0.71\%}_{-1.68\%}$  respectively.\par
The systematic uncertainties due to flavor tagging are generally caused by the 
difference between templates from MC prediction and templates in data. 
To evaluate such difference, delicate flavor tagging commissioning and calibration 
are demanded. These work can be done with the $Z-pole$ data set, which include the 
quark pair production with statistics up to $~10^{10}$. The precision of flavor 
tagging commissioning will be limited by the statistic uncertainty of the 
$Z-$pole hadronic decay data set. We assume the uncertainty of the template would be 
10 times of the statistic uncertainty of the $Z-pole$ hadronic data set, and fluctuate 
the templates according to $Z-pole$ data set statistics uncertainty and fit to the dataset with fluctuated templates. The standard deviations of the fitted signal strength are taken as flavor tagging systematic uncertainty.\par

The non-uniformity of individual higgs decay channels mainly comes from leptonic or 
semi-leptonic higgs decay, which happens with diboson as intermediate states. Since 
there are extra isolation leptons, there are higher chance for these events to have 
at least two isolation leptons. On the other hand, the final states requires at 
least two particles as jet cadidiates, which will reduce the chance of leptonic process with neutrinos in the final states to pass the selection. 
Since the efficiency difference of the signal events and inclusive $\llh$ are mainly due to $H\to WW^*$ and $H\to ZZ^*$ events according to table\ref{tab:uniformity}, the uncertainty was estimated by compare the inclusive higgs efficiency with over estimated and under estimated $H\to WW^*/ZZ^*$ fraction by 5\%.\par
The systematic uncertainty estimation are summarized in table \ref{tab:systematic_uncertainties}.\par
\begin{table}
\label{tab:systematic_uncertainties}
\centering
\begin{tabular}{c|c|c|c}\hline 
             &  $H\to \bpair$ &   $H\to\cpair$    &  $H\to\gpair$    \\ \hline
   modeling  &     0.246\%    &      4.55\%       &     0.27\%       \\ \hline
 $H\to WW^*$ simulation
             & \tabincell{c}{-0.04\% \\ +0.01\%} 
                              &    \tabincell{c}{+3.7\% \\ -3.8\%}  
                                                  & \tabincell{c}{+3.9\% \\ -4.0\%} \\ \hline
 $H\to ZZ^*$ simulation       
             & \tabincell{c}{-0.02\% \\ +0.03\%}   
                              &     0.35\%        &   \tabincell{c}{-1.0\% \\ -0.83\%}\\ \hline
   other background           &   \tabincell{c}{-0.11\% \\ +0.02\%} 
                              &   \tabincell{c}{0.0\%\\ +0.9\%}
                              &   \tabincell{c}{-2.56\%\\ +2.87\%}   \\ \hline           
   extra lepton veto
             &     0.0002\%   &     0.0001\%      &    0.0008\%       \\ \hline
   Jets PFO Multiplicity    
             &     0.0001\%   &     0.0002\%      &    0.0008\%       \\ \hline
   $\cos\theta_{\mathrm{jets}}$ cut
             &     0.0006\%    &    0.0006\%      &    0.006\%        \\ \hline
   $y$ cut
             &     0.0004\%    &    0.0005\%      &    0.007\%         \\ \hline
   Jet pair mass  cut
             &  \tabincell{c}{+0.68\% \\ -0.20\%}
                              &  \tabincell{c}{+0.43\% \\ -1.08\%}
                                                  &   \tabincell{c}{+0.71\% \\ -1.68\%}\\ \hline
      \bpair  -template
             &    0.0048\%    &        0.093\%    &       0.020\%    \\ \hline 
      \cpair -template
             &    0.0016\%    &        0.056\%    &       0.021\%    \\ \hline
      \gpair  -template
             &    0.0045\%    &        0.17\%     &       0.090\%     \\ \hline
      light quark jet template
             &    0.00046\%   &       0.0076\%    &       0.0070\%    \\ \hline   
    Non uniformity
             &    \multicolumn{3}{c}{0.016\%}              \\ \hline  
  Combined   &     \tabincell{c}{+0.73\% \\ -0.36\%}
                              &  \tabincell{c}{+6.1\% \\ -6.2\%}  
                                                  &       \tabincell{c}{+7.6\% \\ -7.7\%}   \\ \hline   
 
\end{tabular} 
\caption{Systematic uncertainties of $H\to \bpair$, $H\to\cpair$ and $H\to\gpair$.}
\end{table}


%\input{texfiles/uncertainties/statistic.tex}
%\input{texfile}

\section{Conclusion}\label{sec:summary}
The $ZH\to \mupair + \bpair/\cpair/\gpair$ brach ratio measurement is done with 5000 \ifb $\elpair$ collsion at $\sqrt{s} = $ 250 \GeV in CEPC experiment, demonstrating the capability of the CEPC experiment in Higgs yukawa coupling with its excellent PFA and flavor tagging performance. 
The statistic uncertainty for $H\to \bpair$, $H\to\cpair$ and $H\to\gpair$ branch fraction is esitmated as 1.1\%, 10.5\% and 5.4\%, while the systematic uncertainty of the 3 channels are $^{+0.73\%}_{-0.36\%}$, $^{+6.1\%}_{-6.2\%}$ and $^{7.6\%}_{-7.7\%}$. This is the first time to study $ZH\to \mupair + \bpair/\cpair/\gpair$ with fully 
simulation sample as well as taking into account the systematic unceraties. The results are consistent with the corresponding results in CEPC-SppC pre-CDR\cite{CEPC_preCDR} in terms of the statistic uncertainties. The results in this document will be incorporated with 
the study of $H\to/\bpair/\cpair/\gpair$ from other channels, like $qqH$ and $\nu\nu H$.\par
%The combination template fit from \eeh,\mmh,\nnh and \qqh $ $evaluate the uncertainty of $H\rightarrow \bpair$, $H\rightarrow \cpair$ and $H\rightarrow \gpair$ to be 0.27\%, 3.2\% and 1.6\%, reflecting the statistic uncertainty with 5000 \ifb integral luminosity data taken at $\sqrt{s} = $250 \GeV at CEPC. These results are done with all the backgrounds and signals from full simulation, and is consistent to the number estimated in pre-CDR. The precision of $H\rightarrow \bpair$ is mainly constrained by \qqh $ $channel, while the other two hadronic decay modes are mainly constrained by \nnh$ $ channel. In \qqh$ $ channel, $H\rightarrow \gpair/\cpair$ suffered from huge backgrounds from hadronic diboson process, and the mis-combination of jet pair degenerate the percision. To solve these problem, it is necessary to optimize the detector and reconstruction performance, which future work should concentrate on.

\clearpage
\pagebreak
\newpage

%\input{texfiles/appendix}

\bibliographystyle{cepcBibStyleWoTitle}
\bibliography{instructions}

\end{document}
